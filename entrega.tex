\documentclass{homeworg}

\AtBeginDocument{\title{Entrega, Geometría Lineal}
\author{\hspace{\halte}\hspace{-4.02em}\normalsize Benaroya~Garzas, Isidro\\
\hspace{\halte}\normalsize Carpes~Martínez, Antonio~Alberto\\
\hspace{\halte}\hspace{-2.74em}\normalsize Muela~Cascallana, Juan~José\\
\hspace{\halte}\hspace{-5.35em}\normalsize Salamanca~Camacho, Jaime\\
\hspace{\halte}\hspace{1.68em}\normalsize Wu, Xiaoye}}
\usepackage{physics}
\usepackage{xcolor}
\usepackage{enumitem}
\usepackage{systeme}
\usepackage{tikz}
\RequirePackage[spanish]{cleveref}
\definecolor{mygreen}{HTML}{446600}
\definecolor{myblue}{cmyk}{1,.72,0,0.3}
\definecolor{myred}{cmyk}{0,.91,.79,.56}

\renewcommand{\bar}[1]{\overline{#1}}
\DeclareMathAlphabet{\pazocal}{OMS}{zplm}{m}{n}
\DeclareSymbolFont{bbold}{U}{bbold}{m}{n}
\DeclareSymbolFontAlphabet{\mathbbold}{bbold}
%\DeclareMathSizes{5}{5}{4.2}{4}

\renewcommand{\thefootnote}{\roman{footnote}}

\def\firstcircle{(0,0) circle (1.5cm)}
\def\secondcircle{(0:2cm) circle (1.5cm)}

%\colorlet{line}{myblue}

    

%\everymath\expandafter{\the\everymath \color{MidnightBlue}}
%\everydisplay\expandafter{\the\everydisplay \color{myblue}}

\newcommand{\rot}[1]{\text{rot}({#1})}
\newcommand{\myblue}{\color{myblue}}
\newcommand{\dive}[1]{\text{div}({#1})}
\makeatletter
\newcommand*{\transpose}{%
  {\mathpalette\@transpose{}}%
}
\newcommand*{\@transpose}[2]{%
  % #1: math style
  % #2: unused
  \raisebox{\depth}{$\m@th#1\intercal$}%
}
\makeatother


\renewcommand{\dot}[1]{
  \accentset{\mbox{\bfseries .}}{#1}}
%\renewcommand{\vec}[1]{{\color{MidnightBlue}\bold{#1}}}
\newcommand{\bfemph}[1]{\textbf{\emph{#1}}}
\newcommand{\hlb}[1]{{\sethlcolor{LightBlue}\hl{#1}}}
\newcommand{\partials}[2]{\frac{\partial #1}{\partial #2}}
\newcommand*\then[1][\text{}]{\mathrel{\stackrel{\makebox[0pt]{\mbox{\normalfont\tiny #1}}}{\Rightarrow}}}
\newcommand{\smol}[1]{{\fontsize{4}{5}\selectfont #1 }}
\newcommand{\ssmol}[1]{{\fontsize{3}{4}\selectfont #1 }}
\newcommand{\R}{\mathbb{R}}  
\newcommand{\V}{\texttt{V}}  
\newcommand{\Z}{\mathbb{Z}}
\renewcommand{\d}{\text{d}}
\newcommand{\C}{\mathbb{C}}
\newcommand{\N}{\mathbb{N}}
\newcommand{\Q}{\mathbb{Q}}
\newcommand{\K}{\mathbb{K}}
\newcommand{\Ss}{\pazocal{S}}
\newcommand{\omegab}[0]{\color{myblue}{\vect{\upomega}}}
\newcommand{\pbrac}[2]{\{#1,#2\}}
\newcommand{\tqp}[0]{(t,\vec{q},\vec{p})}
\newcommand{\dif}[2]{\frac{\text{d} #1}{\text{d} #2}}
\newcommand{\diff}[1]{\Big(\frac{\text{d} #1}{\text{d}t}\Big)_{\text{f}}}
\newcommand{\difm}[1]{\Big(\frac{\text{d} #1}{\text{d}t}\Big)_{\text{m}}}
\newcommand{\vect}[1]{\boldsymbol{\mathbf{#1}}}
\newcommand{\ext}{^{(\text{e})}}
\newcommand{\ls}{\ \ \ \ }
\renewcommand{\epsilon}{\varepsilon}
\renewcommand{\phi}{\varphi}
\renewcommand{\mathring}[1]{\accentset{\circ}#1}
\newcommand{\interior}[1]{\accentset{\circ}#1}


% ===============================================
% MATH
% ===============================================
\newcommand*\tq{\text{ tal que }}
\newcommand*\con{\ |\ }
\newcommand*\Sea{\text{Sea }}
\newcommand*\sea{\text{sea }}
\newcommand*\grado{\text{deg}}
\newcommand*\maxim[1]{\text{m\'ax}(#1)}
\newcommand*\minim[1]{\text{m\'in}(#1)}
\newcommand*\minimb[1]{\text{m\'in}\{#1\}}
\newcommand*\mcd[1]{\text{mcd}(#1)}
\newcommand*\mcdb[1]{\text{mcd}\{#1\}}
\newcommand*\mcm[1]{\text{mcm}(#1)}
\newcommand*\mcmb[1]{\text{mcm}\{#1\}}
\newcommand{\doubleline}[2]{$\left\{
                \begin{array}{ll}
                  #1\\
                  #2\\
                \end{array}
              \right.$
  }
\newcommand{\mdoubleline}[2]{\left\{
                \begin{array}{ll}
                  #1\\
                  #2\\
                \end{array}
              \right.
  }
\newcommand*\oops{        {\small{\text{\underline{ABSURDO}}}}}
\newcommand*\ok{        {\underline{OK}}}
\newcommand\eqq{\mathrel{\stackrel{\makebox[0pt]{\mbox{\normalfont\tiny ?}}}{=}}}

\newcommand\textlesser[1]{\mathrel{\stackrel{\makebox[0pt]{\mbox{\normalfont\tiny #1}}}{<}}}
\newcommand\textto[1]{\mathrel{\stackrel{\makebox[0pt]{\mbox{\normalfont\tiny #1}}}{\longrightarrow}}}
\newcommand\inq{\mathrel{\stackrel{\makebox[0pt]{\mbox{\normalfont\tiny ?}}}{\in}}}



\newcommand*{\spaces}{\hspace{1em}}



% ===============================================
% Integration
% ===============================================
\newcommand{\rst}[1]{\big|
_{#1}}
\newcommand{\intl}{\int\limits}
\renewcommand{\P}{\mathbb{P}}
\newcommand*\upper[1]{\mathscr{U}(#1)}
\newcommand*\lowerr[1]{\mathscr{L}(#1)}
\renewcommand*\H{\mathscr{H}}
\newcommand*\vol[1]{\text{vol}(#1)}
\newcommand*\sumall[1]{\sum_{#1 =1}}
\newcommand*\Sumall[1]{\sum\limits_{#1 =1}}
\newcommand*\toinfty{^{\infty}}
\newcommand*\minimm[1]{\text{m}_{#1}}
\newcommand*\maximm[1]{\text{M}_{#1}}
\newcommand*\alphlist[1]{
\begin{enumerate}[label=\textbf{(\alph*)}]
#1
\end{enumerate}
}
\newcommand*\defb[1]{\{#1\}}
\newcommand*\deffb[1]{\Bigg\{#1\Bigg\}}
\newcommand*\tvec[2]{t_{#1;#2}=\partials{\Phi_{#1}}{#2}}
\newcommand{\compl}{\mathsf{c}}
\newcommand{\hmidthen}{\hmid \then \hmid}


\renewcommand{\figurename}{Fig.}



\newcommand{\bola}[1]{\text{B}(#1)}
\newcommand{\bolac}[1]{\bar{\text{B}}(#1)}

\newcommand{\ep}{\text{exp}}

\newcommand{\ord}[1]{\text{ord}\ #1}
\newcommand{\hmid}{\hspace{3em}}
\newcommand{\hsma}{\hspace{1em}}
\newcommand{\pa}{\texttt{Péndulos acoplados.xlsx}}
\newcommand{\mig}{\texttt{ Momento Inercia Giroscopo.xlsx }}


\newcommand{\ninfty}{\ \overset{n\to \infty}{\longrightarrow}\ }
\newcommand{\rtn}[1]{\sqrt[\leftroot{-3}\uproot{3}n]{#1}}
\newcommand{\rtnp}[1]{\sqrt[\leftroot{-3}\uproot{3}n]{_{p}#1}}
\newcommand{\rtNp}[1]{\sqrt[\leftroot{-3}\uproot{3}N]{_{p}#1}}
\newcommand{\rtf}[2]{\sqrt[\leftroot{-3}\uproot{3}#1]{#2}}
\newcommand{\rtff}[3]{\sqrt[\leftroot{-3}\uproot{3}#1]{_{#3}#2}}
\newcommand{\clim}[2]{\lim\limits_{\substack{#1 \\ #2}}}
\newcommand{\Lim}[1]{\lim\limits_{#1}}
\newcommand{\impeq}{\overset{\text{i}}{\equiv}}
\newcommand{\peq}{\overset{\text{p}}{\equiv}}
\renewcommand{\L}{\mathscr{L}}
\renewcommand{\and}{\ \wedge\ }
\renewcommand{\eqref}[1]{[\ref{#1}]}
\newcommand{\halte}{2.2em}
\newcommand{\rg}{\text{rg}}

\newcommand{\xto}[2][->]{
% correct vertical setting by egreg:
% http://tex.stackexchange.com/a/59660/13304
\tikz[baseline=-\the\dimexpr\fontdimen22\textfont2\relax]{
\node[anchor=south,font=\scriptsize, inner ysep=1.5pt,outer xsep=5pt](x){#2};
\draw[thick,shorten <=2pt,shorten >=3.4pt,dashed,#1](x.south west)--(x.south east);
}
}



\usepackage{scalerel,stackengine}
\stackMath
\newcommand\hatt[1]{%
\savestack{\tmpbox}{\stretchto{%
  \scaleto{%
    \scalerel*[\widthof{\ensuremath{#1}}]{\kern.1pt\mathchar"0362\kern.1pt}%
    {\rule{0ex}{\textheight}}%WIDTH-LIMITED CIRCUMFLEX
  }{\textheight}% 
}{2.4ex}}%
\stackon[-6.9pt]{#1}{\tmpbox}%
}
\parskip 1ex


\usepackage{xstring}

\def\typesystem#1{%
    \begingroup\expandarg
    %\baselineskip=1.5\baselineskip% 1.5 to enlarge vertical space between lines
    \StrSubstitute{\noexpand#1}+{&+&}[\tempsystem]%
    \StrSubstitute\tempsystem={&=&}[\tempsystem]%
    \StrSubstitute\tempsystem,{\noexpand\cr}[\tempsystem]%
    $\vcenter{\halign{&$\hfil\strut##$&${}##{}$\cr\tempsystem\crcr}}$%
    \endgroup
}


\begin{document}
\renewcommand{\proofname}{Solución}

\maketitle

\normalsize
\exercise
Sean $L:=\qty{x_{1}=0}$ y $$X:=\qty{\qty[x_{0}:x_{1}:x_{2}]\in\P^{2}:x_{0}^{4}x_{2}-x_{1}^{5}+x_{1}x_{0}^{4}+x_{0}^{5}=0}$$
\begin{enumerate}[label=\roman*)]
\item Determinar si $X$ es una variedad proyectiva.
\item Determinar si la intersección $L\cap X$ es una subvariedad proyectiva y en caso negativo, calcular $\V\qty(L\cap X)$.
\end{enumerate}

\begin{proof}
\begin{enumerate}[label=\roman*)]
\item[]
\item Tomamos los siguientes puntos:
$$P_{0}=\qty[0:0:1],\hmid P_{1}=\qty[-1:1:1];$$
se puede comprobar con facilidad que $P_{0},P_{1}\in X$, por lo que si $X$ fuese una variedad proyectiva, se tendría que $V=\V\qty(\qty{P_{0},P_{1}})\subset X$.

$V$, al ser una recta, está descrita por una ecuación de la forma:
$$V:\alpha_{0}x_{0}+\alpha_{1}x_{1}+\alpha_{2}x_{2}=0,$$sustituyendo los puntos $P_{i}$, tenemos el siguiente sistema de $\alpha_{i}$:
\begin{equation*}
\left\{
\begin{split}
\alpha_{0}&=\alpha_{1},\\
\alpha_{2}&=0;
\end{split}
\right.\hmidthen V=\qty{x_{0}+x_{1}=0}
\end{equation*}
Podemos ver que el punto $Q=[2:-2:3]\in V$, pero no satisface la ecuación de $X$, por lo que $V\not\subset X$, y consecuentemente se determina que $X$ no es una variedad proyectiva.

\item La intersección $W=L\cap X$ está descrita por la siguiente ecuación:
\begin{equation}\label{eq:1.1}
W:x_{0}^{4}x_{2}+x_{0}^{5}=0,
\end{equation}
Despejando la ecuación \eqref{eq:1.1}, $W$ sólo contiene dos puntos en $\P^{2}$:
$$P_{2}=[0:0:1],\hmid P_{3}=[1:0:-1]$$
Calculamos por tanto $Z=\V\qty(W)=\V\qty(\qty{P_{2},P_{3}})$; sea la ecuación:
$$Z:\beta_{0}x_{0}+\beta_{1}x_{1}+\beta_{2}x_{2}=0,$$
Sustituyendo los puntos $P_{2}$ y $P_{3}$, tenemos que:
$$\left\{
\begin{split}
0=\beta_{0}&=\beta_{2},\\
\beta_{1}&\in\K;
\end{split}
\right.\hmidthen Z=L=\qty{x_{1}=0}$$
Tenemos por tanto $\V\qty(L\cap X)=L$, y como $L\backslash (L\cap X)$ no es vacío\footnote{Por ejemplo, $Q=[2:0:1]\in L\backslash (L\cap X)$.}, $L\cap X$ no es una subvariedad proyectiva.

\end{enumerate}
\end{proof}

\exercise
Consideramos en $\P^{3}$ punto $P:=\qty[0:0:1:2]$ y las rectas:
\begin{equation*}
L_{1}:=\systeme{x_{0}+x_{1}+x_{2}+x_{3}=0;, 
        x_{2}-x_{3}=0;}\hmid \text{y}\hmid L_{2}:=\systeme{x_{1}+x_{2}+x_{3}=0;,
        x_{2}+x_{3}=0.}
\end{equation*}

\begin{enumerate}[label=\roman*)]
\item Exhibir ecuaciones implícitas de todas las rectas que pasan por el punto $P$ y cortan a la recta $L_{1}$.
\item Exhibir ecuaciones implícitas de todas las rectas que pasan por el punto $P$ y cortan a la recta $L_{2}$.
\item Exhibir ecuaciones implícitas de todas las rectas que pasan por $P$ y cortan a $L_{1}$ y a $L_{2}$. Calcular en dichos casos $L\cap L_{1}$ y $L\cap L_{2}$.
\end{enumerate}

\begin{proof}\myblue

\begin{enumerate}[label=\roman*)]
\item[]
\item En primer lugar obtendremos la expresión de un punto genérico $P_1$ de la recta $L_1$ a través de sus ecuaciones paramétricas. Estas ecuaciones las podemos obtener tomando dos puntos cualesquiera de la recta. En nuestro caso valdrían los puntos $$Q_1=\left[1:1:-1:-1\right]\hmid\text{y}\hmid Q_2=\left[ 1:-1:0:0\right].$$
    
    Considerando la proyección canónica del espacio vectorial al proyectivo $\pi :\K^4 \to \P^3$, $u\mapsto \left[u \right]$, podemos definir los vectores de $\K^4$ asociados a los puntos $Q_i$ como $$u_1:=(1,1,-1,-1)\hmid \text{y}\hmid u_2:=(1,-1,0,0).$$
    
    El subespacio vectorial asociado a la recta proyectiva $L_1$ vendrá dado por esos dos vectores, que son linealmente independientes $$\hatt{L_1}=\pi ^{-1}(L_1)\cup \{0\}=L(\{u_1,u_2 \}).$$ Por tanto, unas ecuaciones paramétricas del plano vectorial $\hatt{L_1}$ con respecto a la base estándar de $\K^4$ son:
    \begin{equation*}
     \hatt{L_{1}}:\left\{\begin{aligned} x_{0}& =\\x_1 & = \\x_2 & =\\x_3 & =\\\end{aligned}\sysdelim. . \systeme[rst]{\lambda +\mu , {-}\lambda  + \mu , {-} \mu, {-} \mu}\right. \hmid \text{con} \hmid  (\lambda,\mu ) \in \K^2\end{equation*}
     Por lo que las ecuaciones paramétricas de $L_1$ respecto de la referencia estándar de $\P^3$ son:
     \begin{equation*}
{L_{1}}:\left\{\begin{aligned} x_{0}& =\\x_1 & = \\x_2 & =\\x_3 & =\\\end{aligned}\sysdelim. . \systeme[rst]{\lambda +\mu , {-}\lambda  + \mu , {-} \mu, {-} \mu}\right. \hmid \text{con} \hmid  [\lambda:\mu] \in \P^1
     \end{equation*}
     
     En consecuencia, tenemos que el punto genérico de $L_1$ es $P_1=\left[ \mu+\lambda:\mu-\lambda:-\mu:-\mu \right] $ con $\left[ \lambda:\mu \right] \in \P^1$. Por tanto todas las rectas $L_1'$ que pasen por el punto $P$ y corten a la recta $L_1$ pasarán también por el punto $P_1$. Por tanto ya sabemos que $L_1'=\V\qty(\qty{P,P_1})$ por ser los dos puntos independientes y podemos obtener fácilmente unas ecuaciones implícitas:
     \begin{equation*}
         L_1': \rg\begin{pmatrix*}[r] x_0 & \lambda+\mu & 0 \\ x_1 & -\lambda+\mu & 0 \\ x_2 & -\mu & 1 \\ x_2 & -\mu & 2 \end{pmatrix*}=2
         \hmidthen L_{1}':\systeme[x_{0}x_{1}x_{2}x_{3}]{{(\mu-\lambda)} x_{0}-{(\mu+\lambda)} x_{1}=0;,-{\mu} x_{1}-{2(\mu-\lambda)} x_{2}+{(\mu-\lambda)}x_{3}=0;}
     \end{equation*}
     con la restricción $[\lambda:\mu]\in\P^{1}$.
     
     \clearpage
     
     \item Para este segundo apartado seguiremos un procedimiento completamente análogo al anterior. Tomamos dos puntos cualesquiera $R_1,R_2 \in L_2$ de forma que $L_2=\V(\{R_1,R_2\})$. 
     
     Por ejemplo podemos escoger $$R_1=\left[1:0:0:0\right]\hmid\text{y}\hmid R_2=\left[0:2:-1:1\right].$$ 
     
     De esta forma unas ecuaciones paramétricas de la recta $L_2$ con respecto a la referencia estándar serían:
         \begin{equation*}
            L_2:\left\{\begin{aligned} x_{0}& =\\x_1 & = \\x_2 & =\\x_3 & =\\\end{aligned}\sysdelim. . \systeme[rst]{\alpha , 2\beta, {-}\beta, \beta}\right.\hmid \text{con}\hmid \qty[\alpha:\beta] \in \P^1.
         \end{equation*}
   Por tanto, podemos escribir el punto genérico de $L_2$ como $$P_2=\qty[\alpha: 2\beta: -\beta: \beta]\hmid\text{con}\hmid\qty[ \alpha:\beta] \in \P^1.$$ Siguiendo el mismo razonamiento que en el apartado anterior sabemos que la recta $L_2'$ que pasa por el punto $P$ y corta a $L_2$ también pasa por el punto $P_2$. Como estos dos puntos son independientes sabemos que $L_2'=\V(\{P,P_2\})$ y por tanto podemos obtener fácilmente unas ecuaciones implícitas:
\begin{equation*}
         L_2': \rg\begin{pmatrix*}[r]
         x_0 & \alpha & 0 \\ 
         x_1 & 2\beta & 0 \\ 
         x_2 & -\beta & 1 \\ 
         x_2 & \beta & 2 \end{pmatrix*}=2 \hmidthen L_2': \systeme{2\beta x_0 -\alpha x_1=0, -3\beta x_1-4\beta x_2+2\beta x_3=0},\hmid \text{con}\hmid \qty[\alpha:\beta]\in\P^{1}.
     \end{equation*}

\item La recta $L$ que verifica estas condiciones es un caso particular de la recta $L_1'$ y además un caso particular de la recta $L_2'$ calculadas en los apartados anteriores. Por lo tanto tendrá que cumplir las ecuaciones de $L_1'$ y de $L_2'$ simultáneamente para un cierto $\qty[\lambda:\mu] \in \P^1$ y para un $\qty[\alpha:\beta] \in \P^1$. A la vista de nuestras ecuaciones implícitas simplemente basta con igual coeficientes para cada coordenada (siempre salvo proporcionalidad). Por tanto,
    \begin{equation*}
        \systeme[\mu \lambda \alpha \beta]{\mu - \lambda -2\beta =  0 , \mu + \lambda -\alpha = 0 , \mu -3\beta = 0 , 2\mu - 2\lambda -4\beta=  0} \hmidthen \systeme{\lambda=\beta,\mu=3\beta,\alpha=4\beta,\beta=\beta}
    \end{equation*}
    Sustituyendo todo en función de $\beta$ en las ecuaciones de $L_2'$ obtenemos las ecuaciones implícitas de $L$:
    \begin{equation*}
           L: \systeme{2\beta x_0 -4\beta x_1    =0, -3\beta x_1-4\beta x_2+2\beta x_3=0} \hmid\text{con}\hmid \qty[\alpha:\beta] \in \P^1.
     \end{equation*}
     
     Como tenemos que $\beta \neq 0$ podemos suponer $\beta = -1$, por lo que $\left[ \alpha:\beta \right]=\qty[\frac{\alpha}{-\beta}:-1 ] \in \P^1$. Con esto podemos reescribir las ecuaciones implícitas de $L$ como
         \begin{equation*}
           L: \systeme{- x_0 +2 x_1    =0,  3 x_1+4 x_2-2 x_3=0.}
     \end{equation*}
     Para calcular $L \cap L_1$ simplemente tenemos que concatenar las ecuaciones y eliminar una ecuación redundante:
      \begin{equation*}
           L \cap L_1: \systeme{x_0 + x_1 +  x_2  +  x_3=0,    x_2 -  x_3=0, - x_0 +2 x_1    =0}
           \hmidthen           L \cap L_1: \left\{\begin{aligned} x_{0}& =\\x_1 & = \\x_2 & =\\x_3 & =\\\end{aligned}\sysdelim. . \systeme[rst]{2\gamma, \gamma, {-}\frac{3}{2}\gamma, {-}\frac{3}{2}\gamma}\right.
     \end{equation*}
Tomando $\gamma=2$ puesto que $x_1 \neq 0$ tenemos que $L \cap L_1:=\qty{[ 4:2:-3:-3]}$.

Para calcular $L \cap L_2$ seguimos un proceso completamente análogo al caso anterior:
 \begin{equation*}
           L \cap L_2: \systeme{  x_1 +  x_2  -  x_3=0 ,    x_2 +  x_3=0 , - x_0 +2 x_1    =0}
           \hmidthen
           L \cap L_2: \left\{\begin{aligned} x_{0}& =\\x_1 & = \\x_2 & =\\x_3 & =\\\end{aligned}\sysdelim. . \systeme[rst]{2\rho, \rho, {-}\frac{1}{2}\rho, \frac{1}{2}\rho}\right.
     \end{equation*}
Tomando $\rho=2$ puesto que $x_1 \neq 0$ y tenemos que $L \cap L_2:=\qty{\qty[4:2:-1:1]}$. 



\end{enumerate}

\end{proof}


\exercise
Consideremos en $\P^{3}$ los puntos $A:=\qty[1:0:0:0]$ y $B:=\qty[0:1:0:0]$ y las rectas $$L_{1}:=\qty{x_{0}=x_{1},\ x_{2}=0}\hmid \text{y}\hmid L_{2}:=\qty{x_{0}+x_{1}=0,\ x_{3}=0}.$$Sea $f:\P^{3}\xto{} P^{3}$ una aplicación proyectiva que transforma $A$ en $B$ y deja fijos todos los puntos de $L_{1}\cup L_{2}$.

\begin{enumerate}[label=\roman*)]
\item Probar que $L_{1}$ y $L_{2}$ no son coplanarias.
\item Calcular la familia de los planos que contienen a la recta $L_{i},\ i=1,2$.
\item Demostrar que los planos que contienen a las rectas $L_{i}$ son invariantes por $f$ para $i=1,2$.
\item ?`Es $f$ una homografía??`Es única?
\item Hallar la matriz respecto de la referencia proyectiva estándar de $f:\P^{3}\xto{} \P^{3}$.
\end{enumerate}

\end{document}